%\documentclass[landscape]{slides}
%\usepackage{amssymb,amsmath,amsthm}
%\pagestyle{empty}

\usepackage{datetime}
\documentclass{article}
\usepackage[utf8]{inputenc}
\usepackage{fancyhdr}
\pagestyle{fancy}
\lhead{CRI Report Generator}
\lfoot{UIC Center for Research Informatics}
\rhead{page \thepage}
\rfoot{ \today}
\cfoot{}
\renewcommand{\headrulewidth}{0.4pt}
\renewcommand{\footrulewidth}{0.4pt}
\usepackage{tikz}
\usetikzlibrary{shapes.misc}

\tikzset{cross/.style={cross out, draw=black, minimum size=2*(#1-\pgflinewidth), inner sep=0pt, outer sep=0pt},
%default radius will be 1pt. 
cross/.default={1pt}}


\usepackage{listings}
\usepackage{color}
\usepackage{graphicx}
\usepackage{wrapfig}

\definecolor{dkgreen}{rgb}{0,0.6,0}
\definecolor{gray}{rgb}{0.5,0.5,0.5}
\definecolor{mauve}{rgb}{0.58,0,0.82}

\lstset{
  language=Java,
  aboveskip=3mm,
  belowskip=3mm,
  showstringspaces=false,
  columns=flexible,
  basicstyle={\small\ttfamily},
  numbers=none,
  numberstyle=\tiny\color{gray},
  keywordstyle=\color{blue},
  commentstyle=\color{dkgreen},
  stringstyle=\color{mauve},
  breaklines=true,
  breakatwhitespace=true,
  tabsize=3
}
 
\usepackage{titling}
\setlength{\droptitle}{-2cm}
%\pagestyle{headings}
\title{CRI Report Generator}
\author{Gobind Singh}
\date{ }
 
\begin{document}
 
\maketitle
 
\tableofcontents
 
\section{Requirements}

\indent The Center for Research Informatics (CRI) requires reports, which summarize its process, results, and analysis, to be authored in a simplified manner.  The solution should fullfill the following requirements:
\begin{enumerate}
    \item The information specified in static config files(relating to analysis methods and reference databases) should be output to the final report, without additional inputs requirements in the authored report.
    \item Alerts for the report author, validating set memberships of analysis techniques and their methods, in cases such as mispecification or spelling error.
    \item Figures, which can link to a variety of formats such as image(jpg,png,pdf), data(bam,fastq), and tables(csv), should be available.
    \item Formating relating to table of contents with links, numbered citations, and sections styled with logo, title, and date
\end{enumerate}

%\pagebreak 
\section{Design Overview}
\\
\begin{wrapfigure}{r}{0.25\textwidth} %this figure will be at the right
    \vspace{-20}
    \centering
%    \includegraphics[lineheight=5,width=0.25\textwidth]
    \includegraphics[width=2cm,height=3cm,keepaspectratio]
{process}
\end{wrapfigure}
\\
Extensible Stylesheet Language Transformations (XSLT) will be used to transform Extensible Markup Language(XML) documents to the required form.  The XML documents will consist of configuration files and the authored report file.  An XSLT stylesheet specifies a report's format and structure; a cascading stylesheet specifies design and style; and a document schema specifies valid document relationships.  

\\
XML is a free, platform neutral, human-readable data transfer protocol.  Although it is larger and more verbose than other formats, such as JSON or binary, it provides better portability.  XSLT is domain specific to XML and is capable of using XPath functions to select DOM elements.  XSLT is the ideal solution for static html generation from xml and configuration files.  An XML schema can be created to allow authors to write the report content in a simplified manner.  Further, the xml schema can be validated by xslt, which can assert errors.  Finally, CSS separates the style from the document data, thus providing clearer design.   


% \includegraphics[scale=.5]{process}

%\begin{tikzpicture}
%\filldraw[black] (0,2.5) circle (1.5pt) node[anchor=west] {config files};
%\filldraw[black] (0,2) circle (1.5pt) node[anchor=west] {schema};
%\filldraw[black] (0,1.5) circle (1.5pt) node[anchor=west] {authored xml report};
%
%\draw[very thick,->] (3.75,2.2) -- (4.75,2.2) node[anchor=west] {XSLT};
%
%\draw[blue, very thick] (7,0.5)rectangle (8.2,2.2);
%\filldraw[black] (7,2.5) circle (1.5pt) node[anchor=west] {html output};
%\end{tikzpicture}


\addcontentsline{toc}{section}{\protect\numberline{}Scope}%
\section*{Scope}
XSLT was not designed for use in web applications.  To serve html files, a standard web solution should be found.  XSLT can be used at both the client side or server side, and does not require javascript to be enabled.  XSLT solutions are interoperable with web frameworks such as Django and Rails. 

\section{Implementation}
The following are xml examples:
\begin{itemize}
\item config file
\begin{lstlisting}
<?xml version="1.0" encoding="UTF-8"?>
<cfg>
<analysis id="NGS" title="Quality control report of FASTQC data">
    <method id="QC">
        <citation>http://www.bioinformatics.com/etc/</citation>
    </method>
    <description>We evaluate raw FASTQ to determine qualit issues with each library.
    </description>
</analysis>
<analysis id="genome_alignment" title="Align to ref genome">
    <method id="bowtie2">
        <citation>Langmead et al Fast gapped align 2012</citation>
        <description>ref genome aligner</description>
    </method>
</analysis>
</cfg>
\end{lstlisting}
\item authored xml file
\begin{lstlisting}
<?xml version="1.0" encoding="UTF-8"?>
   <report>
   <title>Variant calling and annotation for Dr.Smith</title>
   <date>July 9,1985</date>
   <analysis id="NGS">
       <method id="QC"></method>
       </analysis>
       <analysis id="genome_alignment">
               <method id="Trim"></method>
               </analysis>     
               </report>
\end{lstlisting}

\item XSLT uses DOM traversal to select elements.  The context returned by a single selected element is then acted upon by its most specific matching template.
\begin{lstlisting}
<?xml version="1.0" encoding="UTF-8"?>
<xsl:stylesheet xmlns:xsl="http://www.w3.org/1999/XSL/Transform"
    xmlns:xs="http://www.w3.org/2001/XMLSchema"
    exclude-result-prefixes="xs"
    version="2.0">
    <xsl:template match="/">
        <html>
            <head>
                <h2>Report for Dr Smith</h2>
                <div><xsl:value-of  select="current-date()"/></div><br/>
            </head>
            <body>
                <ul>
                <div>
                    <xsl:apply-templates select="/report/analysis"/>
                    <xsl:apply-templates select="/report/analysis/method"/>
                </div>
                </ul>
            </body>
        </html>
    </xsl:template>
    
    <xsl:template match="analysis">
        <h3>
            <xsl:apply-templates select="@id"/>
        </h3>  
        <li>
            <ul><xsl:apply-templates select="/report/analysis/method/@id"></xsl:apply-templates></ul>
            <!-- <xsl:apply-templates select="{name()}"/> -->
            <ul><xsl:apply-templates select="document('contentconfig.xml')/cfg/analysis/method[@id= current()/@id]/citation"/></ul>
        </li>
    </xsl:template>
    <!--<xsl:template match="analysis/method[@id='QC']">
    <li>
            <xsl:apply-templates select="document('contentconfig.xml')/cfg/analysis/method[@id = 'QC']/citation"/>
            --><!--<xsl:apply-templates select="document('contentconfig.xml')/analysis/method/{@id}/citation"/>-->
    

    <xsl:template match="method">
        <li>
            <xsl:apply-templates select="document('contentconfig.xml')/cfg/analysis/method[@id= current()/@id]/citation"/>
        </li>
    </xsl:template>
    
</xsl:stylesheet>

\end{lstlisting}

\item Relax NG, a schema language for xml, can be used to validate authored XML files.  For more complicated validation, such as asserting set membership, Schematron, an xml schema language which finds tree patterns in parsed documents, can be used. 

\item Editors for XSLT come in both free and non-free versions.  Free xslt editors such as Treebeard (http://treebeard.sourceforge.net/) and non-free Oxygen (http://www.oxygenxml.com/) both support the Saxon processor, which provides XSLT, XPath, and XML Schema support.   

Configuration files will be written in xml; as will the authored report files.

\item CSS, cascading style sheets, is the standard for styling of sections. 


\end{itemize}


\end{document}


